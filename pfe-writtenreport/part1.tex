\selectlanguage{english}
\part{Internship Description}

\chapter{Work Description}

\section{Internship first part: ENSTA}

\section{Internship second part: CEA}

\subsection{CEA background and history}

Le CEA (Commissariat à l’Energie Atomique) a été créé en 1948, selon la volonté du général De Gaulle, qui voulait assurer à la France une autonomie énergétique.  C’est selon cette volonté qu’a été, pendant de nombreuses années, orienté le développement du CEA. 
Cette mission fut un grand succès au regard de la place qu’occupe aujourd’hui le nucléaire dans le secteur énergétique français. Cependant, le secteur énergétique n’est pas le seul domaine dans lequel le nucléaire occupe une place importante. Le CEA a ainsi participé au développement de tous les domaines utilisant le nucléaire (imagerie médicale, études fondamentales sur les particules, etc…)

Au fil des années, le CEA a souhaité diversifier ses activités, cherchant à rester en permanence à la pointe des découvertes scientifiques.  C’est ainsi qu’ont été créés plusieurs laboratoires de recherche.
Le premier de ces laboratoires à avoir été créé est le LETI (Laboratoire d’Electronique et de Technologie de l’Informatique) en 1967. Suivront, dans les années 2000, deux autres laboratoires : le le LIST et le LITEN. Ces trois laboratoires constituent le CEA Tech dont je parle plus loin dans ce document.
	Ces laboratoires permettent au CEA de développer de nouvelles technologies, sans se limiter au seul domaine du nucléaire. On peut également citer plusieurs partenariats qui ont été mis en place et qui soulignent bien la volonté du CEA d’être partout à la fois dans le domaine scientifique. On trouve ainsi le LSCE (le Laboratoire des Sciences du Climat et de l’Environnement) qui a été mis en place avec le CNRS, Neurospin,  un centre de neuro-imagerie, le CNS et le CNG (respectivement Centre National de Séquençage et Centre National de Génotypage), le MIRcen (Molecular Imaging research center), HelioBiotech (recherche sur les biocarburants), Nanno-Innov (robotique)
	
\subsection{CEA current scenario}

Le CEA d’aujourd’hui est un EPIC (Etablissement Public à caractère Industriel et Commercial). Le siège social est basé à Saclay et possède différents sites en France (une dizaine). Le CEA est divisé en plusieurs grandes catégories : le pôle défense (DAM), le pôle nucléaire (DEN), le pôle recherches technologiques (DRT), le pôle science de la matière (DSM) et le pôle science du vivant (DSV) (cf organigramme en annexe).
Pour ma part, j’ai effectué mon stage au CEA Tech, qui est une structure un peu particulière du CEA. Le but de cette unité est de permettre un lien entre la recherche fondamentale et l’industrie. Le CEA Tech fonctionne donc sur les principes d’une entreprise. J’ai été rattachée à la DRT pour la durée de mon stage.
Le CEA emploie plus de 16000 personnes sur toute la France. Chaque année le CEA fonctionne grâce à un budget de 4,4 milliards d’euros.

\subsection{Work environment}

Les trois laboratoires dont j’ai parlé plus haut forment donc le CEA Tech. On compte donc :
-	Le LETI, ‘’qui concentre ses activités sur les micros et nanotechnologies, ainsi que de leur intégration dans les systèmes’’ (d’après http://www-leti.cea.fr/fr)
-	Le LITEN : le Laboratoire d’Innovation pour les Technologies et des Energies Nouvelles et les nanomatériaux
-	Le LIST qui est  ‘’un institut public de recherche spécialisé dans la conception des systèmes numériques’’ (d’après http://www-list.cea.fr/)

Je ne développerai que ce dernier laboratoire puisque c’est dans celui-ci que j’ai effectué mon stage. Le LIST est lui-même découpé en plusieurs sous-parties. J’ai pour ma part été rattachée au LRI (Laboratoire de Robotique Interactive), sur le site de DIGITEO MOULON à Saclay. Sur ce site il y a environ une quarantaine de personnes comprenant des ingénieurs, des doctorants et des post-doc.

Le LIST fonctionne comme une entreprise, c’est-à-dire que les ingénieurs sont poussés par ‘’la culture du résultat, avec des objectifs de coûts, de délais et de performance’’.  Ainsi le LIST établit chaque année de nombreux partenariats industriels, chaque année il collabore ainsi avec une cinquantaine de grands groupes, ainsi qu’une cinquantaine de PME. Et ce laboratoire connait chaque année un chiffre d’affaire en croissance.

De plus, sont créées chaque année, des start-up (en  douzaine en dix ans) qui constituent l’utilisateur final des innovations portées par le CEA Tech. Ces start-up restent ensuite en contact avec les CEA et le LIST pour continuer à innover et ‘’contribuent ainsi à un écosystème d’innovation en plein essor’’.

\chapter{Work Description}
\lipsum[1-3]
