%realize theirs tasks in human-occupied environments. 

%One basic requirement for such robots are is the capacity of move beteween two 
%configurations 
%Special problems/constraints arise when dealing with multi robot systems. 
%Chosing 

%The increasing use of autonomous mobile robots in humam environment %in a 
%supply chain
% is notorious.
%Specially importance are being given to robot systems that are consisted of 
%muliple robots working in colaboration.
%In this context...

%All mobile robots consisting of a solid block in motion are flat 
%systems~\cite{Defoort2007a}.
%
%A flat system presents the property that states and inputs can be written in 
%terms of the flat outputs $z$ and its derivatives.
%Thus, all system behavior can be expressed by the flat outputs and a finite 
%number of its derivatives ($l$).
%
%TODO reference Veeraklaew et al. in [106] firstly combines the concepts of 
%differential flatness and sequential quadratic programming.
%
%TODO discussion about how to find the mapping $(q, u) \rightarrow z$ (look up 
%Millan 2003).
%
%The approach proposed by Defoort/Milam (and others) takes advantage of the 
%flatness property and search a solution for
%the nonholonomic motion planning problem in the flat space rather than in the 
%configuration space of the mobile robot.
%
%In addition, their approach use B-splines for representing the solution in the 
%flat space. This provide a small local support (TODO verify and elaborate) able 
%to represent
%complex trajectories.
%
%Finally a trajectory optimization routine is done that accounts for all 
%constraints (nonholonomic, geometric and bounded-input) finding an appropriate
%solution.
%
%This approach such as just described assume full knowledge of the environment 
%where the mobile robot is to execute its motion. Defoort adapts this
%method to a sliding window architecture where the motion planning problem is 
%devised throughout time as the mobile robot evolves in its environment and
%discover it.
%%By relaxing the final state constraints, changing the objective function of 
%the NLP and planning for a fixed timespan $T_p$ his new planner
%%dynamically generates the trajectory as the robot moves.
%
%Furthermore, another adaptation of this method is done in \cite{Defoort2007a} 
%for multi robot systems. Thanks to an exchange of information
%among the different robots they can adapt theirs trajectories to avoid 
%robot-to-robot collisions and loss of communication in a decentralized fashion.


%\section{}

\section{Receding horizon approach}

%As said before we are interest in solving the problem of planning a trajectory 
%for nonholonomic mobile robots.

Let us consider the problem of planning a trajectory for a unique mobile robot
to go from an initial configuration $q_{initial}$ to a final one $q_{final}$ while
respecting kinematic, dynamic and geometric constraints.




